% !TeX root=main.tex
% در این فایل، عنوان پایان‌نامه، مشخصات خود و چکیده پایان‌نامه را به انگلیسی، وارد کنید.

%%%%%%%%%%%%%%%%%%%%%%%%%%%%%%%%%%%%
\baselineskip=.6cm
\begin{latin}
\latinuniversity{Iran University of Science and Technology}
\latinfaculty{Computer Engineering Department}
\latinsubject{Computer Engineering }
\latinfield{Artificial Intelligence}
\latintitle{Stance Detection for Textual Content in Social Media}
\firstlatinsupervisor{Dr. Sayyed Sauleh Eetemadi}

\latinname{Ghazaleh}
\latinsurname{Mahmoudi}
\latinthesisdate{February  2024}
\latinkeywords{Stance Detection for Textual Content in Social Media}
\en-abstract{
Nowadays, social media is a platform for freely expressing and sharing opinions and thoughts. This leads to the fact that by analyzing the data available on social media, a broad and comprehensive perspective on various users’ opinions and sides about different topics could be gained. These topics include political, economic, social, and cultural issues. In Natural Language Processing, stance detection is the process of automatically recognizing the side and stance of a given text about a
specific target. 
\newline
In natural language processing tasks, the way text data is preprocessed significantly affects the performance of the trained model. In this research, seven different levels of preprocessing are introduced and examined. Additionally, to find the architecture of the stance detection model, the idea of Neural Architecture Search (NAS) was inspired. In this method, the model architecture is divided into four main parts, a search space is defined for each part, and adaptive search algorithms are used to design the final architecture. The best proposed model ultimately utilizes BERTweet as the encoder and a CNN classifier. The proposed architecture achieved an F1-Score of 74.47\%, showing a 19.97\% improvement over the Baseline model. Furthermore, the proposed method ranked third among 19 participants in a climate change stance detection event. Additionally, due to the lack of training data for different topics, stance detection without training data was also investigated. This approach, which uses large language models and prompt engineering, introduces four approaches based on different prompt types. Then, the performance of the proposed prompts was compared with other methods for stance detection without training data. The introduced approach achieved an F1-Score of 57.33\%, showing a 2.03\% improvement over similar approaches.}
\latinfirstPage
\end{latin}
