% !TeX root=main.tex

\university{علم و صنعت ایران}

\faculty{دانشکده مهندسی کامپیوتر}

\department{گروه هوش مصنوعی}

\subject{مهندسی کامپیوتر}

\field{هوش مصنوعی و رباتیک}

\title{تشخيص موضع متنی در شبکه‌های اجتماعی}

\firstsupervisor{دکتر سید صالح اعتمادی}

\name{غزاله}

\surname{محمودی}

\studentID{400722156}

\thesisdate{بهمن 1402}

\firstPage
\besmPage
\davaranPage

\vspace{.5cm}

%\renewcommand{\arraystretch}{1.2}
\begin{center}
	\begin{tabular}{| p{8mm} | p{18mm} | p{.17\textwidth} |p{14mm}|p{.2\textwidth}|c|}

		\hline
		ردیف	& سمت & نام و نام خانوادگی & مرتبه \newline دانشگاهی &	دانشگاه یا مؤسسه &	امضـــــــــــــا\\
		\hline
		۱  &	استاد راهنما & دکتر سید \newline  صالح اعتمادی & استادیار & دانشگاه \newline علم و صنعت ایران &  \\
		\hline
				۲  &	استاد \newline مدعو داخلی & دکتر  \newline  حسین رحمانی & استادیار & دانشگاه \newline علم و صنعت ایران &  \\
		
		\hline
				۳ &	 استاد \newline مدعو‌‌خارجی& دکتر  \newline سعیده ممتازی & دانشیار & دانشگاه امیرکبیر &  \\
		\hline

	\end{tabular}
\end{center}


%%%%%%%%%%%%%%%%%%%%%%

%\iffalse
\esalatPage
\mojavezPage
\newpage
\iffalse
\thispagestyle{empty}
\centerline{\Large \titlefont  تقـــدیم }
\begin{center}
	محل قرار گرفتن متن قـدرانی و تقدیم در نــسخه نهایی پایان‌نامه. 
\end{center}
\fi

% -- متن سپاس‌گزاری
\begin{acknowledgementpage}
سپاس خداوندگار حکیم را که با لطف بی کران خود، آدمی را زیور عقل آراست و به ما توفیق حرکت در مسیر کسب علم را ارزانی داشت. در آغاز وظیفه خود می‌دانم از زحمات استاد گران‌قدر و فرزانه جناب آقای دکتر سید صالح اعتمادی که راهنمایی اینجانب را در دوره کارشناسی ارشد عهده‌دار بودند، بی‌نهایت سپاس‌گزار هستم. همچنین از آقای بابک بهکام‌کیا که به عنوان همکار در بخشی از پژوهش حضور داشتند کمال تشکر را دارم. در آخر از خانواده عزیزم به پاس حمایت‌های بی دریغ و عاطفه سرشارشان در طول این مدت کمال تشکر و قدردانی را دارم.
	

	% با استفاده از دستور زیر، امضای شما، به طور خودکار، درج می‌شود.
	\signature 
\end{acknowledgementpage}

%\fi
%%%%%%%%%%%%%%%%%%%%%%

%امروزه شبکە‌های اجتماعی بستری جهت آزادانه بیان کردن و به اشتراک گذاشتن عقاید و نظرات می‌باشد. 
\keywords{تشخیص موضع، پردازش زبان طبیعی، یادگیری عمیق، مدل‌های زبانی بزرگ}
\fa-abstract{
امروزه شبکه‌های اجتماعی بستری جهت بیان آزادانه عقاید و اشتراک گذاشتن نظرات می‌باشد. این موضوع سبب شده است که با تحلیل دادە‌های موجود در شبکه‌های اجتماعی بتوان دید وسیع و جامعی از موضع‌ کاربران متفاوت نسبت به موضوعات مختلف به دست آورد. از جمله این موضوعات می‌توان به مسائل سیاسی، اقتصادی، اجتماعی و فرهنگی اشاره کرد. در پردازش زبان طبیعی، به فرآیند تشخیص خودکار موضع متن نسبت به موضوعی مشخص و معین، تشخیص موضع گفته می‌شود. 
\newline
در مسائل پردازش زبان طبیعی از جمله تشخیص موضع، نحوه پیش‌پردازش داده‌های متنی در عملکرد مدل آموزش دیده تاثیر به سزایی دارد. در این پژوهش هفت سطح مختلف پیش‌پردازش معرفی شده و مورد بررسی قرار می‌گیرد. علاوه بر این، برای یافتن معماری مدل تشخیص موضع، از ایده جستجو معماری عصبی (\lr{NAS}) الهام گرفته شد. در این روش با تقسیم معماری مدل به چهار بخش اصلی و تعریف فضای جستجو برای هر بخش و استفاده از الگوریتم‌ جستجو تطبیقی، معماری نهایی طراحی می‌شود. در نهایت بهترین مدل پیشنهاد شده از کدگذار
\lr{BERTweet}
و رده‌بند 
\lr{CNN}
استفاده می‌کند. 
معماری طراحی شده توانست به
$74.47$
درصد در معیار 
\lr{F1} 
دست یابد و نسبت به مدل پایه
$19.97$
درصد بهبود داشته باشد. همچنین روش ارائه شده رتبه سوم را بین 19 شرکت‌کننده رویداد تشخیص موضع در تغییرات اقلیمی کسب کرد. 
از سوی دیگر با توجه به کمبود داده‌های آموزشی برای موضوعات متفاوت، تشخیص موضع بدون داده آموزشی نیز مورد بررسی قرار گرفت. در این روش که از مدل‌های زبانی بزرگ و مهندسی پرامپت استفاده می‌کند، چهار رویکرد بر اساس انواع مختلف پرامپت معرفی شد. سپس عملکرد پرامپت‌های پیشنهادی با سایر روش‌های تشخیص موضع بدون داده آموزشی مورد مقایسه قرار گرفت. رویکرد معرفی شده توانست به مقدار 
$57.33$
درصد در معیار 
\lr{F1} 
دست یابد و نسبت به رویکرد‌های مشابه
$2.03$
درصد بهبود داشته باشد.}
\abstractPage
\newpage\clearpage

